Il principio alla base delle tecniche PET consiste nella rilevazione simultanea di due raggi \textGamma generati da un evento di annichilazione \textit{elettrone-positrone}, generato da iniezione di un tracciante radioattivo in un soggetto. I positroni sono emessi dal nucleo di isotopi instabili e ricchi di protoni, durante il processo di decadimento radioattivo \cite{Bailey2014}. Infatti, questi isotopi acquisiscono la stabilità tramite un processo di decadimento che converte un protone in un neutrone, con la generazione di un positrone. Il positrone, chiamato anche \textit{antielettrone}, è l'antiparticella dell'elettrone. Infatti, esso ha la stessa massa e spin 1/2 dell'elettrone ma presenta una carica elettrica \textit{+e}. Più precisamente, il processo di decadimento che si verifica è di tipo \textit{beta plus} ($\beta^+$) in cui un protone legato ($p$) del nucleo dell'isotopo radioattivo si trasforma in un neutrone legato, un positrone ($e^+$) e un neutrino ($\nu_e$) \cite{Betaplus}. Il processo può essere riassunto dalla equazione:
\begin{equation}
	p\to n + e^+ + \nu_e.
\end{equation}
I \textit{tracer} radioattivi utilizzati nelle applicazione PET sono analoghi delle comuni molecole biologiche, come il glucosio, peptidi e proteine, in cui l'isotopo radioattivo viene utilizzato per sostituire un costituente della molecola. Per esempio,