\chapter*{Introduzione}
\addcontentsline{toc}{chapter}{Introduzione}
La diagnostica per immagini si riferisce a generici processi non invasivi attraverso i quali è possibile ottenere una rappresentazione grafica delle strutture interne o di processi fisiologici di un soggetto. Le immagini prodotte possono essere utilizzati da medici (o recentemente anche da algoritmi di machine learning) per ottenere informazioni mediche utili per la diagnostica di alcune malattie. Esistono diverse tecniche di imaging tra cui la tomografia a emissione di positroni (PET), la tomografia computerizzata a raggi X (CT), la risonanza magnetica (MRI) e imaging a ultrasuoni (UI). Tra queste tecniche, la PET costituisce uno strumento molto potente che permette di acquisire informazioni fisiologiche con un'alta sensitività a differenti processi metabolici a livello delle molecole. Ad esempio, essa viene impiegata per l'analisi di masse tumorali e di patologie neurologiche quali l'Alzheimer, e in cardiologia per patologie delle arterie coronarie \cite{Jiang2019}. Inoltre, i sistemi PET permettono l'integrazione con altre tecniche di imaging, come la CT e la MRI, costruendo un sistema ibrido che unisce i vantaggi delle diverse tecniche. Tuttavia, è evidente che è necessario sviluppare opportuni sensori che permettono l'integrazione di queste tecniche. I rivelatori che sono stati inizialmente usati nei sistemi PET sono costituiri dai \textit{photomultiplier tube} (PMT) che tuttavia, a causa della loro fragilità, al loro costo e alla sensibilità alle interferenze magnetiche, hanno spinto la ricerca verso altri tipi di sensori con caratteristiche migliori. La diffusione di sistemi miniaturizzati realizzati in silicio e il continuo miglioramento delle tecniche CMOS hanno portato all'utilizzo e allo studio di sensori quali gli avalanche photodiodes (APD) e i silicon photomultipliers (SiPM). I sensori e le tecniche utilizzate nella tomografia a emissione di positroni costituiscono il focus di questo approfondimento.