\documentclass[
	a4paper,
	cleardoublepage=empty,
	headings=twolinechapter,
	numbers=autoenddot,
]{scrbook}

\usepackage[utf8]{inputenc}
\usepackage[italian]{babel}
\usepackage{import}
\usepackage{todonotes}
\usepackage{color}
\usepackage{rotating}
\usepackage{hyperref}
\usepackage{url}
\usepackage{pdfpages}
\usepackage{siunitx}
\usepackage[italiano]{algorithm2e}
\usepackage[american, siunitx]{circuitikz}
\usepackage{pdflscape}
\usepackage{subfig}
\usetikzlibrary{fit}
\usepackage[euler]{textgreek}
\usepackage{cite}

\usepackage[format=plain]{caption}

\usepackage{amsmath}
\usepackage{amsfonts}

\usepackage[signatures,swapnames,sans]{frontespizio}

\usepackage{siunitx}
\sisetup{
	per-mode=fraction,
	fraction-function=\tfrac
}

\newcommand{\sub}[1]{\textsubscript{#1}}
\newcommand{\super}[1]{\textsuperscript{#1}}

\newcommand{\Fig}[0]{Fig.}
\newcommand{\Eq}[0]{Eq.}

\begin{document}
	\begin{frontespizio}
		\Margini{3cm}{3cm}{3cm}{3cm}
		\Universita{Bergamo}
		\Logo[43.332mm]{unibg-mark}
		\Divisione{Scuola di Ingegneria}
		\Corso[Laurea Magistrale]{Ingegneria Informatica}
		\Titolo{Sensori per Tomografia a Emissione di Positroni (PET)}
		\Sottotitolo{Corso di Sensori}
		\Punteggiatura{}
		\NRelatore{Prof.}{}
		\Relatore{Gianluca Traversi}
		\Candidato[1057926]{Matteo Verzeroli}
		\Annoaccademico{2022--2023}
		\begin{Preambolo*}
			\usepackage[italian]{babel}
			\usepackage[T1]{fontenc}
			\usepackage[utf8]{inputenc}
			\usepackage{microtype}
			\usepackage{lmodern}
			\graphicspath{{img/}}
			
			\renewcommand{\frontinstitutionfont}{\fontsize{14}{17}\bfseries\scshape}
			\renewcommand{\fronttitlefont}{\fontsize{17}{21}\bfseries\scshape}
			\renewcommand{\frontfootfont}{\fontsize{12}{14}\bfseries\scshape}
		\end{Preambolo*}
	\end{frontespizio}	

	\tableofcontents
	
	\import{./TextFiles/}{Introduzione.tex}
	\import{./TextFiles/}{Fisica della PET.tex}
	\import{./TextFiles/}{Sistema PET.tex}
		
	\bibliographystyle{unsrt}
	\bibliography{./OtherFiles/Bibliography,./OtherFiles/Bibliography_site}
	
	\addcontentsline{toc}{chapter}{Bibliografia}
	
\end{document}